\section{Sorting Lower Bound}

\subsection{Definition}
\begin{itemize}
    \item Viele Sortier-Algoritmen sind vergleichsbasiert
    \item soll eine untere Grenze (Lower Bound) der Laufzeit hergeleitet werden für alle Algorithmen, welche Vergleiche benutzen um n Elemente $x_1, x_2, ..., x_n$ zu sortieren.
\end{itemize}

\subsection{Laufzeit}
\begin{itemize}
    \item Jeder Vergleichs-basierte Sortier-Algorithmus hat eine minimale Laufzeit von: log (n!)
    \item Jeder solch Algorithmus hat somit eine Laufzeit von
    mindestens (gem. Starling-Annäherung): O(n log(n))
\end{itemize}

\subsection{Zählen der Vergleiche}
\begin{itemize}
    \item zählen die Anzahl der Vergleiche
    \item Jeder mögliche Durchgang des Algorithmus korrespondiert mit einem Wurzel-zu-Blatt Pfad ein einem Entscheidungs-Baum
\end{itemize}


\subsection{Höhe des Entscheidungs-Baumes}
\begin{itemize}
    \item Höhe des Entscheidungs-Baumes entspricht der unteren Grenze der Laufzeit
    \item Jede mögliche Input-Permutation führt zu einem anderen Pfad
    \begin{itemize}
        \item wenn dies nicht so wäre, hätte ein Input ...4...5... die selbe Ausgangs-Ordnung wie ...5...4...(was falsch wäre)
    \end{itemize}
    \item Da n!=1*2*...*n Blätter vorhanden sind, beträgt die Höhe mindestens log(n!)
\end{itemize}
\vspace{-8pt}
\begin{center}
    \includegraphics[scale=.2]{graphic/06 SortingLowerBound/höhe.png}
\end{center}
\vspace{-8pt}

\vfill
$ $
\columnbreak
