\section{Graphen}
Ein Graph ist ein Paar (V, E), wobei:
\begin{itemize}
    \item V ist ein Set von Vertizes (Knoten)
    \item E ist eine Collection von Vertizes-Paaren, Kanten (Edge)
    \item Vertizes und Kanten sind Positionen und speichern Elemente
\end{itemize}

\subsection{Kanten-Typen}
\begin{itemize}
    \item gerichtete Kanten
    \begin{itemize}
        \item geordnetes Paar von Vertizes (u,v)
        \item erster Vertex u entspricht dem Ursprung
        \item zweiter Vertex v entspricht dem Ziel
        \item z.B. Flug
    \end{itemize}
    \item ungerichtete Kanten
    \begin{itemize}
        \item ungeordnetes Vertizes-Paar (u,v)
        \item z.B. Flugroute
    \end{itemize}
    \item gerichteter Graph
    \begin{itemize}
        \item alle Kanten sind gerichtet
        \item z.B. Flugplan
    \end{itemize}
    \item ungerichter Graph
    \begin{itemize}
        \item alle Kanten sind ungerichtet
        \item z.B. Flugrouten-Plan
    \end{itemize}
\end{itemize}


\subsection{Laufzeit}
\begin{center}
    \includegraphics[scale=.2]{graphic/11 Graph/Laufzeit.png}
\end{center}
\vspace{-8pt}


\subsection{Terminologie}
\begin{itemize}
    \item Kanten sind \textbf{inzident} (enden) an einem Vertex
    \item \textbf{Adjazente} (benachbarte) Vertizes
    \item \textbf{Grad (Degree)} eines Vertex: Anzahl inzidenter Kanten
    \item \textbf{Parallele Kanten}
    \item \textbf{Schleife}
    \item \textbf{Pfad}
    \begin{itemize}
        \item Sequenz von alternierenden Vertizes und Kanten
        \item beginnt mit einem Vertex
        \item endet mit einem Vertex
        \item jede Kante beginnt und endet an einem ihrer Endpunkte
    \end{itemize}
    \item \textbf{Einfacher Pfad}
    \begin{itemize}
        \item ein Pfad, so dass alle seine Vertizes und Kanten unterschiedlich sind
    \end{itemize}
    \item \textbf{Zyklus}
    \begin{itemize}
        \item zirkuläre Sequenz alternierender Vertizes und Kanten
    \end{itemize}
    \item \textbf{Einfacher Zyklus}
    \begin{itemize}
        \item ein Zyklus, so dass alle seine Vertizes und Kanten unterschiedlich sind
    \end{itemize}
\end{itemize}


\subsection{Eigenschaften}
\begin{itemize}
    \item \textbf{Notation:}
    \begin{itemize}
        \item n = Anzahl Vertizes
        \item m = Anzahl Kanten
        \item deg(v) = Grad von Vertex v
    \end{itemize}
    \item \textbf{Eigenschaft 1:}
    \begin{itemize}
        \item $\Sigma_v deg(v) = 2m$
        \item Beweis: jede Kante wird zweimal gezählt
    \end{itemize}
    \item \textbf{Eigenschaft 2:}
     \begin{itemize}
        \item In einem ungerichteten Graphen ohne Schleifen und ohne parallele Kanten gilt:
        \item m $\leqslant$  n (n - 1) / 2
        \item entspricht einem ungerichteten, einfachen Graphen
        \item Beweis: jeder Vertex besitzt einen Grad von höchstens (n - 1)
     \end{itemize}
\end{itemize}

\vspace{-8pt}
\begin{center}
    \includegraphics[scale=.25]{graphic/11 Graph/Eigenschaften.png}
\end{center}
\vspace{-8pt}


\subsection{Haupt-Methoden}
\begin{center}
    \includegraphics[scale=.25]{graphic/11 Graph/Methoden.png}
\end{center}
\vspace{-8pt}


\subsection{Struktur}

\subsubsection{Kanten-Listen}

\begin{center}
    \includegraphics[scale=.23]{graphic/11 Graph/Kanten-Listen.png}
\end{center}
\vspace{-8pt}

\subsubsection{Adjazenz-Listen}
\begin{center}
    \includegraphics[scale=.23]{graphic/11 Graph/Adjazenz-Listen.png}
\end{center}
\vspace{-8pt}


\subsubsection{Adjazenz-Matrix}
\begin{center}
    \includegraphics[scale=.23]{graphic/11 Graph/Adjazenz-Matrix.png}
\end{center}
\vspace{-8pt}


\subsection{Subgraphen}
\begin{center}
    \includegraphics[scale=.23]{graphic/11 Graph/Subgraphen.png}
\end{center}
\vspace{-8pt}

\subsection{Connectivity}
\begin{center}
    \includegraphics[scale=.2]{graphic/11 Graph/Connectivity.png}
\end{center}
\vspace{-8pt}


\subsection{Bäume und Wälder}
\begin{center}
    \includegraphics[scale=.23]{graphic/11 Graph/Bäume und Wälder1.png}
\end{center}
\vspace{-8pt}


\subsection{Spanning Trees und Wälder}
\begin{center}
    \includegraphics[scale=.25]{graphic/11 Graph/Bäume und Wälder2.png}
\end{center}
\vspace{-8pt}

\newpage